% Options for packages loaded elsewhere
\PassOptionsToPackage{unicode}{hyperref}
\PassOptionsToPackage{hyphens}{url}
%
\documentclass[
]{article}
\usepackage{amsmath,amssymb}
\usepackage{lmodern}
\usepackage{iftex}
\ifPDFTeX
  \usepackage[T1]{fontenc}
  \usepackage[utf8]{inputenc}
  \usepackage{textcomp} % provide euro and other symbols
\else % if luatex or xetex
  \usepackage{unicode-math}
  \defaultfontfeatures{Scale=MatchLowercase}
  \defaultfontfeatures[\rmfamily]{Ligatures=TeX,Scale=1}
\fi
% Use upquote if available, for straight quotes in verbatim environments
\IfFileExists{upquote.sty}{\usepackage{upquote}}{}
\IfFileExists{microtype.sty}{% use microtype if available
  \usepackage[]{microtype}
  \UseMicrotypeSet[protrusion]{basicmath} % disable protrusion for tt fonts
}{}
\makeatletter
\@ifundefined{KOMAClassName}{% if non-KOMA class
  \IfFileExists{parskip.sty}{%
    \usepackage{parskip}
  }{% else
    \setlength{\parindent}{0pt}
    \setlength{\parskip}{6pt plus 2pt minus 1pt}}
}{% if KOMA class
  \KOMAoptions{parskip=half}}
\makeatother
\usepackage{xcolor}
\usepackage[margin=1in]{geometry}
\usepackage{color}
\usepackage{fancyvrb}
\newcommand{\VerbBar}{|}
\newcommand{\VERB}{\Verb[commandchars=\\\{\}]}
\DefineVerbatimEnvironment{Highlighting}{Verbatim}{commandchars=\\\{\}}
% Add ',fontsize=\small' for more characters per line
\usepackage{framed}
\definecolor{shadecolor}{RGB}{248,248,248}
\newenvironment{Shaded}{\begin{snugshade}}{\end{snugshade}}
\newcommand{\AlertTok}[1]{\textcolor[rgb]{0.94,0.16,0.16}{#1}}
\newcommand{\AnnotationTok}[1]{\textcolor[rgb]{0.56,0.35,0.01}{\textbf{\textit{#1}}}}
\newcommand{\AttributeTok}[1]{\textcolor[rgb]{0.77,0.63,0.00}{#1}}
\newcommand{\BaseNTok}[1]{\textcolor[rgb]{0.00,0.00,0.81}{#1}}
\newcommand{\BuiltInTok}[1]{#1}
\newcommand{\CharTok}[1]{\textcolor[rgb]{0.31,0.60,0.02}{#1}}
\newcommand{\CommentTok}[1]{\textcolor[rgb]{0.56,0.35,0.01}{\textit{#1}}}
\newcommand{\CommentVarTok}[1]{\textcolor[rgb]{0.56,0.35,0.01}{\textbf{\textit{#1}}}}
\newcommand{\ConstantTok}[1]{\textcolor[rgb]{0.00,0.00,0.00}{#1}}
\newcommand{\ControlFlowTok}[1]{\textcolor[rgb]{0.13,0.29,0.53}{\textbf{#1}}}
\newcommand{\DataTypeTok}[1]{\textcolor[rgb]{0.13,0.29,0.53}{#1}}
\newcommand{\DecValTok}[1]{\textcolor[rgb]{0.00,0.00,0.81}{#1}}
\newcommand{\DocumentationTok}[1]{\textcolor[rgb]{0.56,0.35,0.01}{\textbf{\textit{#1}}}}
\newcommand{\ErrorTok}[1]{\textcolor[rgb]{0.64,0.00,0.00}{\textbf{#1}}}
\newcommand{\ExtensionTok}[1]{#1}
\newcommand{\FloatTok}[1]{\textcolor[rgb]{0.00,0.00,0.81}{#1}}
\newcommand{\FunctionTok}[1]{\textcolor[rgb]{0.00,0.00,0.00}{#1}}
\newcommand{\ImportTok}[1]{#1}
\newcommand{\InformationTok}[1]{\textcolor[rgb]{0.56,0.35,0.01}{\textbf{\textit{#1}}}}
\newcommand{\KeywordTok}[1]{\textcolor[rgb]{0.13,0.29,0.53}{\textbf{#1}}}
\newcommand{\NormalTok}[1]{#1}
\newcommand{\OperatorTok}[1]{\textcolor[rgb]{0.81,0.36,0.00}{\textbf{#1}}}
\newcommand{\OtherTok}[1]{\textcolor[rgb]{0.56,0.35,0.01}{#1}}
\newcommand{\PreprocessorTok}[1]{\textcolor[rgb]{0.56,0.35,0.01}{\textit{#1}}}
\newcommand{\RegionMarkerTok}[1]{#1}
\newcommand{\SpecialCharTok}[1]{\textcolor[rgb]{0.00,0.00,0.00}{#1}}
\newcommand{\SpecialStringTok}[1]{\textcolor[rgb]{0.31,0.60,0.02}{#1}}
\newcommand{\StringTok}[1]{\textcolor[rgb]{0.31,0.60,0.02}{#1}}
\newcommand{\VariableTok}[1]{\textcolor[rgb]{0.00,0.00,0.00}{#1}}
\newcommand{\VerbatimStringTok}[1]{\textcolor[rgb]{0.31,0.60,0.02}{#1}}
\newcommand{\WarningTok}[1]{\textcolor[rgb]{0.56,0.35,0.01}{\textbf{\textit{#1}}}}
\usepackage{graphicx}
\makeatletter
\def\maxwidth{\ifdim\Gin@nat@width>\linewidth\linewidth\else\Gin@nat@width\fi}
\def\maxheight{\ifdim\Gin@nat@height>\textheight\textheight\else\Gin@nat@height\fi}
\makeatother
% Scale images if necessary, so that they will not overflow the page
% margins by default, and it is still possible to overwrite the defaults
% using explicit options in \includegraphics[width, height, ...]{}
\setkeys{Gin}{width=\maxwidth,height=\maxheight,keepaspectratio}
% Set default figure placement to htbp
\makeatletter
\def\fps@figure{htbp}
\makeatother
\setlength{\emergencystretch}{3em} % prevent overfull lines
\providecommand{\tightlist}{%
  \setlength{\itemsep}{0pt}\setlength{\parskip}{0pt}}
\setcounter{secnumdepth}{-\maxdimen} % remove section numbering
\ifLuaTeX
  \usepackage{selnolig}  % disable illegal ligatures
\fi
\IfFileExists{bookmark.sty}{\usepackage{bookmark}}{\usepackage{hyperref}}
\IfFileExists{xurl.sty}{\usepackage{xurl}}{} % add URL line breaks if available
\urlstyle{same} % disable monospaced font for URLs
\hypersetup{
  pdftitle={Assignment\#2 Carlo, Javier},
  hidelinks,
  pdfcreator={LaTeX via pandoc}}

\title{Assignment\#2 Carlo, Javier}
\author{}
\date{\vspace{-2.5em}2023-04-03}

\begin{document}
\maketitle

\hypertarget{assignment-2}{%
\section{Assignment 2}\label{assignment-2}}

\hypertarget{participants-carlo-broderick-javier-patruxf3n}{%
\subsubsection{Participants: Carlo Broderick, Javier
Patrón}\label{participants-carlo-broderick-javier-patruxf3n}}

\hypertarget{task}{%
\subsection{Task:}\label{task}}

Run your model for the clim.txt data that is posted on Canvas

\hypertarget{data}{%
\subsection{Data:}\label{data}}

\hypertarget{clim.txt-has-the-following-columns-these-4-columns-tell-you-when-climate-observations-were-made}{%
\paragraph{\texorpdfstring{\texttt{clim.txt} has the following columns
these 4 columns tell you when climate observations were
made:}{clim.txt has the following columns these 4 columns tell you when climate observations were made:}}\label{clim.txt-has-the-following-columns-these-4-columns-tell-you-when-climate-observations-were-made}}

\begin{enumerate}
\def\labelenumi{\arabic{enumi}.}
\item
  day
\item
  month
\item
  year
\item
  wy (water year)
\end{enumerate}

\hypertarget{introduction}{%
\subsection{Introduction:}\label{introduction}}

California's perennial crop yields are threatened by the potential
impacts of future climate change, making it critical to develop models
that can project how different climate scenarios may affect crop yields.
The Lobell et al.~2006 model below represents yield anomalies (Y) as a
function of climate variables, including minimum temperature (Tn),
maximum temperature (Tx), and precipitation (P), with the subscript
numbers indicating the month of the variable.

\[Y=−0.015×Tminfeb​​−0.0046×Tminfeb^2​−0.07×Pjan​+0.0043×Pjan^2​+0.28\]

This model is designed to incorporate both climate and crop
uncertainties to provide more accurate projections of future crop yields
under different climate scenarios. The accurate projections provided by
this model can help policymakers and farmers to develop strategies to
mitigate the potential impacts of climate change on California's
perennial crop yields. Below is diagram of how we implemented this model
in R and below the diagram is the code used to take in the provided
climate data and output the almond crop anomalies.

\begin{figure}
\centering
\includegraphics{Almond Diagram.png}
\caption{Conceptual Model}
\end{figure}

\hypertarget{develop-a-function-to-execute-the-model-and-evaluate-it-using-the-clim.txt-dataset.}{%
\subsection{Develop a function to execute the model and evaluate it
using the clim.txt
dataset.}\label{develop-a-function-to-execute-the-model-and-evaluate-it-using-the-clim.txt-dataset.}}

\begin{Shaded}
\begin{Highlighting}[]
\CommentTok{\#load in the packages}
\FunctionTok{library}\NormalTok{(tidyverse)}
\end{Highlighting}
\end{Shaded}

\begin{verbatim}
## -- Attaching packages --------------------------------------- tidyverse 1.3.2 --
## v ggplot2 3.4.0      v purrr   1.0.1 
## v tibble  3.1.8      v dplyr   1.0.10
## v tidyr   1.2.1      v stringr 1.5.0 
## v readr   2.1.3      v forcats 0.5.2 
## -- Conflicts ------------------------------------------ tidyverse_conflicts() --
## x dplyr::filter() masks stats::filter()
## x dplyr::lag()    masks stats::lag()
\end{verbatim}

\begin{Shaded}
\begin{Highlighting}[]
\FunctionTok{library}\NormalTok{(purrr)}

\FunctionTok{library}\NormalTok{(ggpubr)}
\end{Highlighting}
\end{Shaded}

\begin{Shaded}
\begin{Highlighting}[]
\CommentTok{\# Define a function that takes a file path and optional parameter values}
\NormalTok{almond\_model }\OtherTok{\textless{}{-}} \ControlFlowTok{function}\NormalTok{(file, }\AttributeTok{parm1 =} \SpecialCharTok{{-}}\FloatTok{0.015}\NormalTok{, }\AttributeTok{parm2 =} \SpecialCharTok{{-}}\FloatTok{0.0046}\NormalTok{, }\AttributeTok{parm3 =} \SpecialCharTok{{-}}\FloatTok{0.07}\NormalTok{, }\AttributeTok{parm4 =} \FloatTok{0.0043}\NormalTok{, }\AttributeTok{parm5 =} \FloatTok{0.28}\NormalTok{)\{}
  
  \CommentTok{\# Read in the file as a data frame, rename the columns, and remove the first row}
\NormalTok{  clim\_df }\OtherTok{\textless{}{-}} \FunctionTok{read.table}\NormalTok{(file) }\SpecialCharTok{|\textgreater{}} 
  \FunctionTok{rename}\NormalTok{(}\AttributeTok{day =}\NormalTok{ V1, }\AttributeTok{month =}\NormalTok{ V2, }\AttributeTok{year =}\NormalTok{ V3, }\AttributeTok{water\_year =}\NormalTok{ V4, }\AttributeTok{tmax\_c =}\NormalTok{ V5, }\AttributeTok{tmin\_c =}\NormalTok{ V6, }\AttributeTok{precip =}\NormalTok{ V7) }\SpecialCharTok{|\textgreater{}} 
  \FunctionTok{slice}\NormalTok{(}\SpecialCharTok{{-}}\DecValTok{1}\NormalTok{) }\SpecialCharTok{|\textgreater{}} 
    \FunctionTok{mutate\_all}\NormalTok{(as.numeric) }\SpecialCharTok{|\textgreater{}} \CommentTok{\# Convert all columns to numeric}
    \FunctionTok{filter}\NormalTok{(month }\SpecialCharTok{\%in\%} \FunctionTok{c}\NormalTok{(}\DecValTok{1}\NormalTok{,}\DecValTok{2}\NormalTok{)) }\SpecialCharTok{|\textgreater{}} \CommentTok{\# Filter for rows where month is 1 or 2}
    
    \FunctionTok{group\_by}\NormalTok{(year, month) }\SpecialCharTok{|\textgreater{}} \CommentTok{\# Group the data by year and month,}
    \FunctionTok{summarize}\NormalTok{(}\AttributeTok{tmin\_c =} \FunctionTok{min}\NormalTok{(tmin\_c, }\AttributeTok{na.rm =} \ConstantTok{TRUE}\NormalTok{), }\CommentTok{\# Calculate the minimum tmin\_c and total precip for each group}
              \AttributeTok{precip =} \FunctionTok{sum}\NormalTok{(precip, }\AttributeTok{na.rm =} \ConstantTok{TRUE}\NormalTok{)) }\SpecialCharTok{|\textgreater{}} 
    
    \CommentTok{\# Reshape the data from wide to long format}
    \FunctionTok{pivot\_longer}\NormalTok{(}\AttributeTok{cols =} \DecValTok{3}\SpecialCharTok{:}\DecValTok{4}\NormalTok{,}
                 \AttributeTok{names\_to =} \StringTok{"clim\_obs"}\NormalTok{,}
                 \AttributeTok{values\_to =} \StringTok{"value"}\NormalTok{) }\SpecialCharTok{|\textgreater{}} 
    
    \CommentTok{\# Filter for rows where month is 2 and clim\_obs is tmin\_c, or where month is 1 and clim\_obs is precip}
    \FunctionTok{filter}\NormalTok{((month }\SpecialCharTok{==} \DecValTok{2} \SpecialCharTok{\&}\NormalTok{ clim\_obs }\SpecialCharTok{==} \StringTok{"tmin\_c"}\NormalTok{) }\SpecialCharTok{|}\NormalTok{ (month }\SpecialCharTok{==} \DecValTok{1} \SpecialCharTok{\&}\NormalTok{ clim\_obs }\SpecialCharTok{==} \StringTok{"precip"}\NormalTok{)) }\SpecialCharTok{|\textgreater{}} 
    
    \CommentTok{\# Remove the month column}
    \FunctionTok{select}\NormalTok{(}\SpecialCharTok{{-}}\NormalTok{month) }\SpecialCharTok{|\textgreater{}} 
    
    \CommentTok{\# Reshape the data from long to wide format, with columns for tmin\_c and precip}
    \FunctionTok{pivot\_wider}\NormalTok{(}\AttributeTok{names\_from =}\NormalTok{ clim\_obs, }\AttributeTok{values\_from =}\NormalTok{ value) }\SpecialCharTok{|\textgreater{}} 
    
    \CommentTok{\# Calculate the anomaly using the specified parameter values}
    \FunctionTok{mutate}\NormalTok{(}\AttributeTok{anomaly =} \SpecialCharTok{{-}}\NormalTok{parm1 }\SpecialCharTok{*}\NormalTok{ tmin\_c }\SpecialCharTok{+}\NormalTok{ parm2 }\SpecialCharTok{*}\NormalTok{ tmin\_c}\SpecialCharTok{\^{}}\DecValTok{2} \SpecialCharTok{+}\NormalTok{ parm3 }\SpecialCharTok{*}\NormalTok{ precip }\SpecialCharTok{+}\NormalTok{ parm4 }\SpecialCharTok{*}\NormalTok{ precip}\SpecialCharTok{\^{}}\DecValTok{2} \SpecialCharTok{+}\NormalTok{ parm5) }\SpecialCharTok{|\textgreater{}}
    
    \CommentTok{\# Create profit column}
    \FunctionTok{mutate}\NormalTok{(}\AttributeTok{profit =} \DecValTok{100000} \SpecialCharTok{+} \DecValTok{1000}\SpecialCharTok{*}\NormalTok{anomaly)}
    
\NormalTok{    mean\_profit }\OtherTok{=} \FunctionTok{mean}\NormalTok{(clim\_df}\SpecialCharTok{$}\NormalTok{profit)}
  
  \CommentTok{\# Return the final data frame}
  \FunctionTok{return}\NormalTok{(}\FunctionTok{list}\NormalTok{(}\AttributeTok{clim\_df =}\NormalTok{ clim\_df, }\AttributeTok{mean\_profit =}\NormalTok{ mean\_profit))}
  
\NormalTok{\}}
\end{Highlighting}
\end{Shaded}

Test our function!

\begin{Shaded}
\begin{Highlighting}[]
\CommentTok{\# Call the almond\_model function with a file path, and store the result in a new data frame\}}
\NormalTok{clim\_df }\OtherTok{\textless{}{-}} \FunctionTok{almond\_model}\NormalTok{(}\StringTok{"clim.txt"}\NormalTok{)}
\end{Highlighting}
\end{Shaded}

\begin{verbatim}
## `summarise()` has grouped output by 'year'. You can override using the
## `.groups` argument.
\end{verbatim}

\begin{Shaded}
\begin{Highlighting}[]
\CommentTok{\# Print a summary of the anomaly column in the clim\_df data frame}
\NormalTok{data\_frame\_climate }\OtherTok{\textless{}{-}}\NormalTok{ clim\_df[}\DecValTok{1}\NormalTok{]}
\NormalTok{mean\_profit\_value }\OtherTok{\textless{}{-}}\NormalTok{ clim\_df[}\DecValTok{2}\NormalTok{]}
\end{Highlighting}
\end{Shaded}

\begin{Shaded}
\begin{Highlighting}[]
\CommentTok{\# Set up parameter variation distrobutions}
\NormalTok{nsamples }\OtherTok{=} \DecValTok{30}

\NormalTok{parm1 }\OtherTok{=} \FunctionTok{rnorm}\NormalTok{(}\AttributeTok{mean=}\SpecialCharTok{{-}}\FloatTok{0.015}\NormalTok{, }\AttributeTok{sd =} \FloatTok{0.1}\NormalTok{, }\AttributeTok{n=}\NormalTok{nsamples)}
\NormalTok{parm2 }\OtherTok{=} \FunctionTok{rnorm}\NormalTok{(}\AttributeTok{mean=}\SpecialCharTok{{-}}\FloatTok{0.0046}\NormalTok{, }\AttributeTok{sd =} \FloatTok{0.1}\NormalTok{, }\AttributeTok{n=}\NormalTok{nsamples)}

\NormalTok{parms }\OtherTok{=} \FunctionTok{cbind.data.frame}\NormalTok{(parm1, parm2)}

\CommentTok{\# use pmap }
\CommentTok{\# takes function name and then names of all parameters that don\textquotesingle{}t change}
\NormalTok{results }\OtherTok{=}\NormalTok{ parms }\SpecialCharTok{\%\textgreater{}\%} \FunctionTok{pmap}\NormalTok{(almond\_model,}
                         \AttributeTok{file =} \StringTok{"clim.txt"}\NormalTok{,}
                         \AttributeTok{parm3=}\SpecialCharTok{{-}}\FloatTok{0.07}\NormalTok{, }
                         \AttributeTok{parm4=}\FloatTok{0.0043}\NormalTok{, }
                         \AttributeTok{parm5 =} \FloatTok{0.28}\NormalTok{)}
\end{Highlighting}
\end{Shaded}

\begin{verbatim}
## `summarise()` has grouped output by 'year'. You can override using the
## `.groups` argument.
## `summarise()` has grouped output by 'year'. You can override using the
## `.groups` argument.
## `summarise()` has grouped output by 'year'. You can override using the
## `.groups` argument.
## `summarise()` has grouped output by 'year'. You can override using the
## `.groups` argument.
## `summarise()` has grouped output by 'year'. You can override using the
## `.groups` argument.
## `summarise()` has grouped output by 'year'. You can override using the
## `.groups` argument.
## `summarise()` has grouped output by 'year'. You can override using the
## `.groups` argument.
## `summarise()` has grouped output by 'year'. You can override using the
## `.groups` argument.
## `summarise()` has grouped output by 'year'. You can override using the
## `.groups` argument.
## `summarise()` has grouped output by 'year'. You can override using the
## `.groups` argument.
## `summarise()` has grouped output by 'year'. You can override using the
## `.groups` argument.
## `summarise()` has grouped output by 'year'. You can override using the
## `.groups` argument.
## `summarise()` has grouped output by 'year'. You can override using the
## `.groups` argument.
## `summarise()` has grouped output by 'year'. You can override using the
## `.groups` argument.
## `summarise()` has grouped output by 'year'. You can override using the
## `.groups` argument.
## `summarise()` has grouped output by 'year'. You can override using the
## `.groups` argument.
## `summarise()` has grouped output by 'year'. You can override using the
## `.groups` argument.
## `summarise()` has grouped output by 'year'. You can override using the
## `.groups` argument.
## `summarise()` has grouped output by 'year'. You can override using the
## `.groups` argument.
## `summarise()` has grouped output by 'year'. You can override using the
## `.groups` argument.
## `summarise()` has grouped output by 'year'. You can override using the
## `.groups` argument.
## `summarise()` has grouped output by 'year'. You can override using the
## `.groups` argument.
## `summarise()` has grouped output by 'year'. You can override using the
## `.groups` argument.
## `summarise()` has grouped output by 'year'. You can override using the
## `.groups` argument.
## `summarise()` has grouped output by 'year'. You can override using the
## `.groups` argument.
## `summarise()` has grouped output by 'year'. You can override using the
## `.groups` argument.
## `summarise()` has grouped output by 'year'. You can override using the
## `.groups` argument.
## `summarise()` has grouped output by 'year'. You can override using the
## `.groups` argument.
## `summarise()` has grouped output by 'year'. You can override using the
## `.groups` argument.
## `summarise()` has grouped output by 'year'. You can override using the
## `.groups` argument.
\end{verbatim}

\begin{Shaded}
\begin{Highlighting}[]
\CommentTok{\# now we can extract results from the list as above}
\NormalTok{mean\_profit }\OtherTok{=} \FunctionTok{map\_df}\NormalTok{(results,}\StringTok{\textasciigrave{}}\AttributeTok{[}\StringTok{\textasciigrave{}}\NormalTok{, }\FunctionTok{c}\NormalTok{(}\StringTok{"mean\_profit"}\NormalTok{))}

\CommentTok{\# and we can add the parameter values for each run}
\NormalTok{mean\_profit\_w\_params }\OtherTok{=} \FunctionTok{cbind.data.frame}\NormalTok{(mean\_profit, parms)}
\end{Highlighting}
\end{Shaded}

\begin{Shaded}
\begin{Highlighting}[]
\NormalTok{p1 }\OtherTok{=} \FunctionTok{ggplot}\NormalTok{(mean\_profit\_w\_params, }
            \FunctionTok{aes}\NormalTok{(parm1, mean\_profit, }\AttributeTok{col=}\NormalTok{parm2))}\SpecialCharTok{+}\FunctionTok{geom\_point}\NormalTok{(}\AttributeTok{cex=}\DecValTok{2}\NormalTok{)}\SpecialCharTok{+}
  \FunctionTok{labs}\NormalTok{(}\AttributeTok{y=}\StringTok{"Mean Annual Profit USD"}\NormalTok{, }\AttributeTok{x=}\StringTok{"Temperature Paramerter 1"}\NormalTok{)}

\NormalTok{p2 }\OtherTok{=} \FunctionTok{ggplot}\NormalTok{(mean\_profit\_w\_params, }
            \FunctionTok{aes}\NormalTok{(parm2, mean\_profit, }\AttributeTok{col=}\NormalTok{parm1))}\SpecialCharTok{+}\FunctionTok{geom\_point}\NormalTok{(}\AttributeTok{cex=}\DecValTok{2}\NormalTok{)}\SpecialCharTok{+}
  \FunctionTok{labs}\NormalTok{(}\AttributeTok{y=}\StringTok{"Mean Annual Profit USD"}\NormalTok{, }\AttributeTok{x=}\StringTok{"Temperature Paramerter 2"}\NormalTok{)}

\FunctionTok{ggarrange}\NormalTok{(p1,p2)}
\end{Highlighting}
\end{Shaded}

\includegraphics{assignment2_files/figure-latex/unnamed-chunk-6-1.pdf}

In your same groups -

\begin{itemize}
\item
  Develop a profit model for your almond yield (you can make this up -
  think about what the parameters would be)

\begin{verbatim}
* you might assume a baseline profit and then adjust according to the anomaly  
\end{verbatim}

  \begin{itemize}
  \tightlist
  \item
    there are many ways to combine the almond yield and profit
    functions; you can have the profit function ``call''/use the almond
    yield function; or create a wrapper function that calls them in
    sequence (first the almond yield and then the profit function)
  \end{itemize}
\item
  Do a simple informal sensitivity analysis of almond yield profit using
  at least 2 parameters
\item
  Create a single graph of the results - you can decide what is the most
  meaningful graph
\end{itemize}

Submit as a group: a knitted Rmarkdown document that includes your
graph, and your R files for almond yield and profit model

\end{document}
